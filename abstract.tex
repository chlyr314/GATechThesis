\clearpage
%\addcontentsline{toc}{chapter}{Abstract}
\begin{center}
	\textbf{\Large Abstract}\\
	\vspace{\baselineskip}
\end{center}

A novel framework for the structural analysis of planar frames is developed, 
which 
combines tools from finite element analysis, mathematical programming and homotopy 
theory. The core building block in this work is a versatile, high-performance, hybrid 
beam-column element which is formulated based on nonlinear programming 
principles. The 
geometrically-exact kinematic assumptions are adopted, thereby allowing for 
arbitrarily large displacements and rotations, while inelastic behavior is modeled by
assuming a discretization of selected cross-sections into layers or fibers, which can 
incorporate a multiaxial constitutive law but act independently of neighboring layers. 
The element is capable of capturing both geometric and material nonlinearities with 
just one element per structural member while maintaining all the numerically 
attractive properties pertaining to the structure of the resulting global stiffness 
matrix.

The interaction between axial, shear and flexural effects, especially during inelastic 
deformation, is accounted for by incorporating a multiaxial constitutive law at the 
level of cross-section fibers. In particular, a fast and robust return-mapping 
algorithm is developed which is utilizing the zero transverse normal stress assumption 
in order to arrive at a reduced stress space formulation. This allows for a more 
efficient stress update both in terms of memory and computation costs. The 
implementation assumes a $J_2$ von Mises material with combined isotropic and 
kinematic hardening. Such reformulation of general three-dimensional or axisymmetric 
stress update procedures is crucial for beam elements that rely on scarce meshes but 
higher order quadratures in order to achieve accuracy, since elastoplastic analyses by 
fibre-discretized elements increases the computational cost considerably.

The proposed beam element is embedded in a parametric nonlinear programming framework 
which is developed to facilitate analysis considerations beyond response to mechanical 
loading. In particular, we take advantage of the optimization reformulation of the 
underlying variational structure of the mechanics problem and develop a naturally 
parameterized nonlinear programming framework which can easily handle parametric 
investigations, such as design optimization or sensitivity analysis, while also 
providing the theoretical and numerical tools that ensure global convergence 
characteristics, provided certain regularity conditions hold. A predictor-corrector 
type numerical continuation algorithm suited for this framework is also developed 
which can account for any parameterization and derivative discontinuities along the 
solution path.  

To improve the performance of the aforementioned numerical continuation algorithm, a 
reliable and efficient prediction scheme is proposed that is considerably faster than 
conventional approaches that utilize second derivative information. Instead, a 
weighted least squares fitting is carried out at the start of an incremental step, 
provided that a number of previously converged solution points are stored and 
are  available in 
memory. The curve generated by this fitting process is capable to emulate the local 
geometry of the actual solution curve much better since an interpolating condition is 
not enforced. In addition, by using an appropriate weighting function, we can assign 
increased weights to solutions closer to the current step. This leads to a versatile 
scheme that can provide a suite of additional options, as far as general 
predictor-corrector algorithms are concerned and, more importantly, it can be applied 
for any problem which can be solved by such procedures. It can be particularly 
attractive for problems described by dense, non-symmetric Jacobians, especially 
if 
coupled with iterative correction schemes such as conjugate gradients.

The present work is partitioned into four parts: the core hybrid beam element, 
the 
multiaxial constitutive model and return mapping algorithm, the naturally 
parameterized framework for the hybrid element and, finally, the weighted least 
squares predictor scheme. To validate the efficacy and accuracy of the 
procedures associated 
with each individual part developed in this work, several numerical tests are 
presented in the thesis. 