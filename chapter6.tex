\chapter{A Simple GUI for the Hybrid NLP Element}

This chapter introduces a Graphical User Interface (GUI) developed as part of 
this work in order to 
facilitate analyses using the hybrid \acrshort{nlp} element. We detail the 
various features that are implemented using the AppDesigner utility of MATLAB, 
version 2022b. The chapter is divided in seven sections, each representing a 
particular tab of the interface. These seven tabs are : 1) Mesh, 2) Materials, 
3) Sections, 4) Boundary Conditions, 5) Analysis Controls, 6) Solve Job and, 
finally, 7) Post-processing. The first five tabs, some of which can be edited 
independently, belong to the pre-processing phase. Tab 6 calls the hybrid NLP 
solver for the given input in tabs 1-5 and tab seven provides post-processing 
infrormation with respect to the structure response in the form of 
force-displacement plots and  configuration history for all steps. A number of 
tooltips can be revealed by hovering over relevant fields.

\section{The Mesh Tab}

The first tab of the GUI, depicted in Fig. \ref{fig:TAB1_marked}, is concerned 
with nodal and element input. We analyze it in terms of parts A to N:

\begin{itemize}
	\item \textbf{A}: The Mesh tab.
	\item \textbf{B}: Facility used to manually add nodes. Numeric values 
	pertaining to the X (horizontal) and Y (vertical) coordinate axes are used 
	in the corresponding fields to define the coordinates of a node. By 
	clicking on the \textit{Add Node} button, the enumerated node is added to 
	the table with its coordinated also listed. The enumeration is automatic. 
	Attempting to add a node with coordinates (X,Y) that are already defined 
	for another node results in an error, as shown in Fig. 
	\ref{fig:TAB1_node_error}.
	\item \textbf{C}: Facility used to delete a particular node. The 
	\textit{Label} field takes as input the node label already generated and 
	removes it from the table. The enumeration of remaining nodes is updated 
	accordingly. If node deletion is performed after the mesh is generated, the 
	latter is no longer valid and all elements are removed.
	\item \textbf{D}: Success/Error indicator of an action pertaining to node 
	generation, coupled with a text field for relevant message output (e.g. see 
	\ref{fig:TAB1_node_error}).
	\item \textbf{E}: Facility to generate a node set from a \textbf{.txt} or 
	\textbf{.xlsx} file. The format for this input file is shown in Fig. 
	\ref{fig:TAB1_nodesfile}. If a number of nodes have already been added 
	manually, this action replaces all existing nodes. In contrast, generating 
	a nodal set from an input file and then manually adding additional nodes 
	results in appending the previously generated nodal list.
	\item \textbf{F}: Table that lists all generated nodes along with their 
	coordinates.
	\item \textbf{G}: Facility used to manually add a (hybrid \acrshort{nlp}) 
	beam element. Since only one element per member is adquate with the 
	proposed formulation, the only relevant inputs are the start and end node 
	labels, indicated in the relevant fields as \textit{Node} $i$ and 
	\textit{Node} $j$ respectively. Again, adding an element with start and end 
	nodes already defined for an existing element results in an error. In 
	addition, element enumeration is, again, automatic.
	\item \textbf{H}: Element deletion facility. The element label used as 
	input in the relevant field and by clicking the \textit{Delete Element} 
	button results in removal of the particular element from the table. The 
	remainig elements are, again, automatically re-enumerated.
	\item \textbf{I}: Success/Error indicator of an action pertaining to 
	element generation, coupled with a text field for relevant message output.
	\item \textbf{J}: Facility to generate an element partition from a 
	\textbf{.txt} or \textbf{.xlsx} file. The format for this input file is 
	shown in Fig. \ref{fig:TAB1_elementsfile}. If a number of elements have 
	already been added manually, this action replaces all existing elements. In 
	contrast, generating an element partition from an input file and then 
	manually adding additional elements results in appending the previously 
	generated element list.
	\item \textbf{K}:  Table that lists all generated elements along with their 
	start and end nodes.
	\item \textbf{L}: Once clicked, the \textit{Clear Mesh} button resets all 
	existing input in the Mesh tab and the user can start over.
	\item \textbf{M}: GUI axes that dynamically update the existing state of 
	nodal and mesh configuration. Figure \ref{fig:TAB1_mesh_example} shows the 
	Mesh tab when the mesh generation is complete for a simple case of a one 
	bay, one story frame. The nodal labels are places right next to their 
	respective nodes and are typeset in normal font while element labels are 
	approximately placed in the midspan of their corresponding elements and are 
	typeset in italic bold fonts.
	\item \textbf{N}: With the \textit{Load problem} facility, the user can 
	load a problem in its entirety. The appropriate file type is ASCII 
	(extension ".txt") and formatting details for the input file are discussed 
	in Section \ref{INPUTFILE} of this Appendix. Once the input file is 
	imported, the user 
	can go directly to tab \textit{Solve Job} and proceed with solving the 
	problem.
\end{itemize}

\begin{figure}
	\centering
	\includegraphics[scale=0.6]{GUIpics/TAB1/TAB1_marked.png}
	\caption{The \textit{Mesh} tab with relevant facilities A-M marked in red 
	circles 
	and enumerated with capital english letters.}
	\label{fig:TAB1_marked}
\end{figure}

\begin{figure}
	\centering
	\includegraphics[scale=0.6]{GUIpics/TAB1/TAB1_node_error.png}
	\caption{Error when attempting to add a node with coordinates already 
	defined for an existing node.}
	\label{fig:TAB1_node_error}
\end{figure}

\begin{figure}
	\centering
	\includegraphics[scale=0.6]{GUIpics/TAB1/TAB1_nodesfile.png}
	\caption{A .txt file with the required format for nodal input. First column 
	is X coordinate, second is Y coordinate. The $k$-th row represents Node 
	$k$.}
	\label{fig:TAB1_nodesfile}
\end{figure}
\begin{figure}
	\centering
	\includegraphics[scale=0.6]{GUIpics/TAB1/TAB1_elementsfile.png}
	\caption{A .txt file with the required format for element input. First 
	column is start node $i$, second is end node $j$. The $k$-th row represents 
	Element $k$.}
	\label{fig:TAB1_elementsfile}
\end{figure}

\begin{figure}
	\centering
	\includegraphics[scale=0.6]{GUIpics/TAB1/TAB1_mesh_example.png}
	\caption{The \textit{Mesh} tab with a completed mesh.}
	\label{fig:TAB1_mesh_example}
\end{figure}


\section{The Materials Tab}

The \textit{Materials} tab follows the \textit{Mesh} tab and does not have any 
dependencies regarding previous input. Figure FIG outlines all relevant fields 
in this tab, again, separated by red cycles based on their functionality. We 
list and describe items A-I below:

\begin{itemize}
	\item \textbf{A}: The \textit{Materials} tab.
	\item \textbf{B}: Material name field. Takes alphanumeric values. If a 
	material name is not specified, a default name \textbf{Material}\_$\#$ is 
	used, where $\#$ takes values 1,2,3... etc. As an example, see Fig. 
	\ref{fig:TAB2_adding_mat_noname}.
	\item \textbf{C}: Numeric fields pertaining to elastic and plastic 
	properties of the material. In the case of purely elastic analysis, dummy 
	or zero values can be used in the plastic fields, as they are not 
	considered.
	\item \textbf{D}: The \textit{Add} and \textit{Delete} material facilities, 
	along with the Success/Error indicator (green lamp icon). To delete a 
	material defined previously, the user needs to click on it in the 
	\textit{Materials Panel} seen on the left side of the tab (\textbf{Item H}, 
	see below), and then click the \textit{Delete} button.
	\item \textbf{E}: Axes that depict the uniaxial stress-strain law for a 
	defined material. In order to activate it, the user will need to click on 
	the specific material in the \textit{Materials Panel} seen on the left side 
	of the tab (\textbf{Item H}, see below). An max strain 
	$\epsilon_{max}=10\epsilon_y$ is specified purely for plotting purposes. 
	This feature can be seen in Fig. \ref{fig:TAB2_materials_added}.
	\item \textbf{F}: The \textit{Update Material} facility updates the elastic 
	and plastic properties of an existing material. The material to be updated 
	needs to be selected in the \textit{Materials panel} seen on the left side 
	of the tab (\textbf{Item H}, see below). 
	\item \textbf{G}: A \textit{Standard materials} facility, which is 
	introduced to quickly add predefined, standardized materials from the 
	drop-down list. These materials are assumed to be elastic-perfectly plastic 
	but can be modified using the \textit{Update Material} facility. Two 
	materials are available: 1) Structural Steel(S235) and 2) Aluminum 
	Alloy(6062) (see Fig \ref{fig:TAB2_predefined_mats}). In contrast with 
	manually adding materials, duplicate 
	materials from the drop-down are not permitted, as can be seen from Fig. 
	\ref{fig:TAB2_adding_already_predefmat}. 
	\item \textbf{H}: This is the \textit{Materials} panel. All materials 
	defined and added are listed here. The user needs to click on a specific 
	material on the list in order to i) delete it, ii) update its parameters or 
	iii) view its uniaxial stress-strain plot.
	\item \textbf{I}: The elastic and plastic properties of a material selected 
	on the \textit{Materials} panel are shown in this table. The tabulated 
	quantities correspond to the material fields in \textbf{B} as follows:
		i) $E$ represents Young's modulus, ii) $\nu$ represents Poisson's 
		ratio, iii) f\_y represents the yield stress of the material, iv) 
		$H_{iso}$ represents the isotropic hardening modulus and v) $H_{kin}$ 
		represents the kinematic hardening modulus.
\end{itemize}

\begin{figure}
	\centering
	\includegraphics[scale=0.6]{GUIpics/TAB2/TAB2_marked.png}
	\caption{The \textit{Materials} tab with relevant facilities A-M marked in 
	red circles 
		and enumerated with capital english letters.}
	\label{fig:TAB2_marked}
\end{figure}


\begin{figure}
	\centering
	\includegraphics[scale=0.6]{GUIpics/TAB2/TAB2_adding_mat_noname.png}
	\caption{Multiple materials defined without a specified material name.}
	\label{fig:TAB2_adding_mat_noname}
\end{figure}

\begin{figure}
	\centering
	\includegraphics[scale=0.6]{GUIpics/TAB2/TAB2_materials_added.png}
	\caption{Potting of uniaxial stress-strain law of material \textbf{MAT}\_1.}
	\label{fig:TAB2_materials_added}
\end{figure}

\begin{figure}
	\centering
	\includegraphics[scale=0.6]{GUIpics/TAB2/TAB2_predefined_mats.png}
	\caption{Available predefined materials from the \textit{Standard 
	materials} drop-down list.}
	\label{fig:TAB2_predefined_mats}
\end{figure}

\begin{figure}
	\centering
	\includegraphics[scale=0.6]{GUIpics/TAB2/TAB2_adding_already_predefmat.png}
	\caption{Error when trying to add a standard material that was already 
	added previously.}
	\label{fig:TAB2_adding_already_predefmat}
\end{figure}

\section{The Sections tab}

The third tab pertains to cross-section defintion. Cross-sections as objects 
are assigned a name, a shape, numeric values pertaining to the basic dimension 
of the selected shape (e.g. two basic dimensions for rectangular sections). 
Moreover, each section is assigned a material. As a consequence the 
\textit{Sections} tab has a dependency on the \textit{Materials} tab. If a  
material gets deleted, then all sections that were assigned that material get 
deleted as well. 

We now proceed with the description of the essential parts of the 
\textit{Sections} tab, shown in Fig. \ref{fig:TAB3_marked}.

\begin{itemize}
	\item \textbf{A}: The \textit{Sections} tab.
	\item \textbf{B}: Here, the user assign a name and a shape to the section. 
	The name field takes alphanumeric values. If not specified, then the name 
	\textbf{Section\_}$\#$ is assigned to the section, where $\# = 1,2,3, ...$. 
	The available shapes from the drop-down list are i) rectangular and ii) 
	(symmetric) wide flange geometries. 
	\item \textbf{C}: In this section, the user assigns numeric values to the 
	relevant fields that pertain to the basic dimensions of the assigned shape. 
	A rectangular shape has to basic dimensions: height $h$ and width $b$. The 
	wide flange shape has 2 additional shapes: web thickness $t_w$ and flange 
	thickness $t_f$. In addition, the user also specifies which material will 
	be assigned to the section. As can be seen from Fig. 
	\ref{fig:TAB3_section_added}, the section \textbf{Rectangular\_Section} is 
	assigned the material \textbf{MAT\_1} defined previously. The drop-down 
	list is also shown, where all other materials defined in the 
	\textit{Materials} tab are available in the list.
	\item \textbf{D}: Axes that depict the cross-section shape selected from 
	the drop-down list. The basic dimensions are also indicated in each case.
	\item \textbf{E}: In this part, the user assigns sections to elements. 
	Tapered element capabilities are not included in the present code, therefor 
	only one section per element is permitted. A section is selected from the 
	\textit{Cross-Sections} panel, which is located at the top left of the tab 
	(see also \textbf{Item G} below) and then an element has to be selected 
	that will be assigned the section. Another feature for this part is the 
	capability to assign a specific to all elements, which is done by selecting 
	a section and then clicking on the \textit{Assign to all elements} button. 
	Lastly, a Success/Error indicator is included here as well in order to 
	facilitate the assignment process. For elements that are successfully 
	assigned a section, a message will be shown that states the name of the 
	assigned section, as shown in Fig. \ref{fig:TAB3_section_assigned} for 
	\textbf{Element 1} which was assigned the section 
	\textbf{Aluminum\_Section}. In contrast, for elements that have not been 
	assigned a section, when selected in the drop-down list, the indicator will 
	turn red and an appropriate message will be displayed. This can be seen in 
	Fig. \ref{fig:TAB3_NOsection_assigned}.
\end{itemize}

\begin{figure}
	\centering
	\includegraphics[scale=0.6]{GUIpics/TAB3/TAB3_marked.png}
	\caption{The \textit{Sections} tab with all essential parts marked in red 
		cycles and enumerated.}
	\label{fig:TAB3_marked}
\end{figure}

\begin{itemize}
	\item \textbf{F}: Table that lists the basic dimensions for the section 
	currently under selection in the \textit{Cross-Sections} panel (see below).
	\item \textbf{G}: The \textit{Cross-Sections} panel is where all sections 
	created are listed. In order to delete or assign a section to an element, 
	it has to be selected from the panel. 
\end{itemize}
\clearpage
\begin{figure}[t]
	\centering
	\includegraphics[scale=0.5]{GUIpics/TAB3/TAB3_section_added.png}
	\caption{Typical section definition process. A previously defined material 
	\textbf{MAT\_1} is assigned to section \textbf{Rectangular\_Section}.}
	\label{fig:TAB3_section_added}
\end{figure}

\begin{figure}[b]
	\centering
	\includegraphics[scale=0.5]{GUIpics/TAB3/TAB3_section_assigned.png}
	\caption{Successful section assignment for \textbf{Element 1}.}
	\label{fig:TAB3_section_assigned}
\end{figure}
\clearpage
\begin{figure}
	\centering
	\includegraphics[scale=0.6]{GUIpics/TAB3/TAB3_NOsection_assigned.png}
	\caption{Indication that \textbf{Element 2} still needs to be assigned a 
	cross-section.}
	\label{fig:TAB3_NOsection_assigned}
\end{figure}

\section{The Boundary Conditions tab}

In the \textit{Boundary Conditions} tab the user specifies the externally 
applied nodal loads \textit{and} and the nodal support conditions so that rigid 
body modes are excluded. The layout of the tab is shown in Fig. FIG and parts 
from A to F are discussed below:

\begin{itemize}
	\item \textbf{A}: The \textit{Boundary Conditions} tab.
	\item \textbf{B}: Here the user selects a node from the drop-down list at 
	the top to apply an external load at a certain nodal \acrshort{dof}. Once 
	the node is selected, the user can specify the magnitude and sign of the 
	applied force in three available force fields: 1) P1, which is aligned with 
	the global X axis (horizontal direction), 2) P2, which is aligned with the 
	global Y axis (vertical direction) and 3) M, which represents the moment 
	and the direction of the axis which it aligns with is determined by the 
	right-hand rule. In Fig. \ref{fig:TAB4_LOADS} an example is shown where 
	horizontal and vertical loads are applied at nodes 2 and 3.
	\item \textbf{C}: This is the \textit{Nodal Loads} table, where all nodes 
	that are assigned a non-zero external load along at least one nodal 
	\acrshort{dof} are listed (see \ref{fig:TAB4_LOADS}).
	
	\item \textbf{D}: Here, the user specifies supports conditions for certain 
	nodal \acrshort{dof}s. Again, after selecting a node from the drop-down 
	list, the user specifies which of the three available \acrshort{dof}s, U1, 
	U2 and U3, are fixed. There is a one-to-one correspondence between P1-P2-M 
	and U1-U2-U3 as far as direction of application. However, here, one only 
	needs to activate the available boxes in order to constrain the 
	corresponding displacement \acrshort{dof}. Figure \ref{fig:TAB4_SUPPORTS} 
	shows a case where node 1 is assigned a roller support condition (U1 and U2 
	fixed), whereas node 4 is assigned a clamp support condition (U1, U2, U3 
	fixed).
	\item \textbf{E}: Again, the table lists the nodes that have at least one 
	displacement \acrshort{dof} fixed (see Fig. \ref{fig:TAB4_SUPPORTS}).
	
	\item \textbf{F}: With the \textit{Clear} button, the user can erase all 
	input in this tab and start over.
\end{itemize}


\begin{figure}
	\centering
	\includegraphics[scale=0.6]{GUIpics/TAB4/TAB4_marked.png}
	\caption{The \textit{Boundary Conditions} tab with all essential parts 
	marked in red cycles and enumerated.}
	\label{fig:TAB4_marked}
\end{figure}

\begin{figure}
	\centering
	\includegraphics[scale=0.6]{GUIpics/TAB4/TAB4_LOADS.png}
	\caption{Input example for external nodal loading specification.}
	\label{fig:TAB4_LOADS}
\end{figure}

\begin{figure}
	\centering
	\includegraphics[scale=0.6]{GUIpics/TAB4/TAB4_SUPPORTS.png}
	\caption{Input example for nodal support specification.}
		\label{fig:TAB4_SUPPORTS}
\end{figure}

\section{The Analysis Controls tab}

In the \textit{Analysis Controls} tab, the user can specify settings pertaining 
to geometric and constitutive assumptions as well as modify the default 
incremental analysis settings. An option to access advanced features is also 
given, whereby one can select between different continuation solvers and, if 
the Homotopy solver is chosen, the user can further specify details such as 
prediction and correction schemes. The latter gives the option to use the 
\acrshort{wls} predictor. 

The basic layout of the tab is shown in Fig. \ref{fig:TAB5_marked}. The marked 
components are:

\begin{itemize}
	\item \textbf{A}: The \textit{Analysis Controls} tab.
	\item \textbf{B}: The option to enable geometrically-exact kinematic 
	assumptions. When it is, the equilibrium solver used by default is 
	Crisfield's cylindrical arc-length method. For this reason, the user can 
	also specify a step-length, $DS$. 
	\item \textbf{C}: This option pertains to the incorporation of material 
	nonlinearities during the analysis. It is disabled by default. If enabled, 
	the user can also specify the number of cross-section layers. This number 
	is the same for all sections in the structure.
	\item \textbf{D}: Option that pertains to the incorporation shear effects 
	during the analysis. If enabled, which is the default setting, and the 
	plastic analysis option is also enabled, then a quadratic distribution for 
	the shear strains is assumed along the cross-section height.
	\item \textbf{E}: Numerical quadrature options. The user can choose between 
	Gauss-Legendre and Gauss-Lobatto schemes. The latter tends to be more 
	accurate for frame structures since the end-points of the domain are 
	included in the quadrature set.
	\item \textbf{F}: The user sets the order of numerical quadrature. The 
	minimum allowed is 1 and the maximum is 12.
	\item \textbf{G}: Number of incremental steps to be used if the Load 
	Control equilibrium solver is chosen. When the Cylindrical arc-length is 
	used, this number simply divides the external load vector in order to get a 
	base load.
	\item \textbf{H}: Maximum number of steps. It pertains only to continuation 
	solvers, such as Crisfield's Cylindrical arc-length and the Homotopy 
	solver. The solver will stop 1) when the external load is surpassed or 2) 
	when the maximum number of steps is reached.
	\item \textbf{I}: The maximum number of correction iterations to be 
	performed during an incremental step.
	\item \textbf{J}: The tolerance used to test convergence in term of the 
	Euclidian norm of the out-of-balance (residual) force vector.
	\item \textbf{K}: Option to show the advanced analysis settings (see below).
	\item \textbf{L}: Once the user has specified all analysis controls in tab, 
	they have to register the input by clicking on the \textit{Register 
	changes} button.
	\item \textbf{M}: The user can reset all settings in this tab to their 
	defaults by using the \textit{Reset to defaults} button.
	\item \textbf{N}: Success/Error indicator for this tab.
\end{itemize}

We now discuss the advanced settings panels, shown in Fig. 
\ref{fig:TAB5_marked_advanced}, which become visible only when the 
\textit{Advanced settings}(\textbf{K}) option is enabled:

\begin{itemize}
	\item \textbf{O}: Drop-down list (see Fig. \ref{fig:TAB5_solvers}) where 
	the user can select between three different solution schemes: 1) Load 
	Control, 2) Crisfield's arc-length and 3) Homotopy solver. By default, 
	Crisfield's solver is enabled. 
	\item \textbf{P}: Numeric fields pertainig to a) minimum step-length, b) 
	maximum step-length and c) step-length adjustment parameter. The first two 
	limits ensure that the step-length will not become exceedingly small or 
	large. The latter adjust the step-length based on how easily the solution 
	process converged in the last step. It is advised that, in case of plastic 
	analyses, parameter $a$ be set to values between 4 and 8. These three 
	options are not relevant to the Load Control scheme. The values shown in 
	Fig. \ref{fig:TAB5_marked_advanced} are the default ones.
	\item \textbf{Q}: Option to force Crisfield's arc-length and the Homotopy 
	solver to repeat the last incremental step until the reach load is within a 
	specified tolerance of the externally applied load. If this option is not 
	enabled the solver will stop when the base load mulitplied by the load 
	factor simply exceed the external load vector.
	\item \textbf{R}: Option that, when enabled, forces the Homotopy and 
	Crisfield's arc-length solvers to repeat steps that failed to converge 
	within the specified iteration counter, but with halved step-length. The 
	same step will be repeated until convergence or the min$DS$ is reached.
	\item \textbf{S}: The options in the \textit{Homotopy solver settings} 
	become visible only when the Homotopy solver is selected in \textbf{O}. The 
	Predictor option allows the user to choose between i) tangent differential 
	predictor and ii) the \acrshort{wls} predictor. If the latter is chosen, an 
	additional panel of options, specific to this precition scheme, appears.
	\item \textit{T}: Here the user can specify the correction scheme to be 
	used. The three options are: i) Normal Flow\cite{Watson:1989}, ii) 
	Secant-Length\cite{Schwetlick87} and iii) Normal Plane\cite{Riks:1979}.
	
	\item \textbf{U}: If the \textit{WLS predictor} is selected in \textbf{S}, 
	the user can specify which variant of the \acrshort{wls} predictor to use. 
	The available options are: i) \acrshort{wlse}, which is the default, ii) 
	\acrshort{wlst} and iii) \acrshort{wlsit}.
	\item \textbf{V}: This parameter determines the number of load-controlled 
	steps to be performed before the homotopy solver using \acrshort{wls} 
	predictor takes over. A number of simple Newton steps need to be performed 
	in order to generate enough solutions available for the \acrshort{wls} 
	scheme.
	\item \textbf{m}: The degree of polynomial basis used in the \acrshort{wls} 
	fitting. In this version, this remains constant throughout the analysis. 
	Future versions will provide an option to automatically adjust it during 
	the solution process.
	\item \textbf{N0}: The number of previously converged points to consider 
	for the \acrshort{wls} fitting. In this version, this remains constant 
	throughout the analysis. Future versions will provide an option to 
	automatically adjust  it during the solution process.
	\item \textbf{beta}: The weight function parameter, with default value 0.2. 
	If set to 1, then there is no weighting during the least squares fitting. 
	Keep this parameter larger than zero.
\end{itemize}
\begin{figure}
	\centering
	\includegraphics[scale=0.6]{GUIpics/TAB5/TAB5_marked.png}
	\caption{The \textit{Analysis Controls} tab with all essential parts 
		marked in red cycles and enumerated.}
	\label{fig:TAB5_marked}
\end{figure}

\begin{figure}
	\centering
	\includegraphics[scale=0.6]{GUIpics/TAB5/TAB5_marked_advanced.png}
	\caption{The \textit{Analysis Controls} tab with the advanced settings 
	enabled.}
	\label{fig:TAB5_marked_advanced}
\end{figure}

\begin{figure}
	\centering
	\includegraphics[scale=0.6]{GUIpics/TAB5/TAB5_solvers.png}
	\caption{The \textit{Analysis Controls} tab with the advanced settings 
	enabled.}
	\label{fig:TAB5_solvers}
\end{figure}



\section{The Input File}\label{INPUTFILE}



