\chapter{Conclusions and Suggestions for Future Research}

\section{Research Summary and Conclusions}
This research work sought to develop a high-performance, \acrshort{nlp} beam 
element 
and embed it in a general parametric optimization framework that takes 
advantage of 
the element formulation. The proposed element features exact kinematic 
capabilities 
coupled with a robust and fast multiaxial stress update algorithm, thereby 
allowing 
for shear-flexure interaction without introducing kinematic approximations to 
displacement and rotational fields. The naturally parameterized \acrshort{nlp} 
framework provides the capability for general and truly global parametric 
investigations of structural frames. In addition, the numerical continuation 
algorithm 
used in the implementation of the path-tracking procedure is equipped with a 
robust, 
fast and reliable predictor, also as part of this research. This prediction 
scheme is 
based on a local weighted least squares fitting and provides superior 
computational 
performance when compared to other conventional prediction schemes. In 
addition, it is 
a generally applicable algorithm that offers even more considerable advantages 
when 
problems possess dense, non-symmetric Jacobians. Thus, a unique framework for 
static structural analysis is presented herein that offers capabilities beyond 
the conventional load parameterized analyses by utilizing parametric 
optimization principles. Another unique feature pertains to the capabilities 
of the beam element to capture general nonlinear response with just one element 
per member in all cases and settings.

\section{Limitations}

Limitations of the work presented herein are primarily a direct 
result of 
assumptions set initially during the formulation of the relevant parts. As far 
as the 
hybrid element is concerned, the present work is focused on quasi-static planar 
problems that are conservative. In addition, cross-sections are assumed to 
remain 
plane after deformation, even if not perpendicular to the beam centerline. This 
means 
that warping effects are not taken into account. Another issue related to 
cross-section 
kinematics is that simple shear strain 
distribution 
patterns 
were assumed, such as uniform and quadratic, and energetic consistency with the 
exact 
shear strain energy is established through the use of the shear coefficient.

For the multiaxial constitutive model, one limitation pertains to the linear 
kinematic 
hardening rule assumed. This simplification was adopted to ease already 
cumbersome 
derivations 
pertaining to the integration of the rate equations. In addition, the model is 
valid 
for cross-section with materials that follow the von Mises yield 
criterion. 
Extension to include other material constitutive laws is, however, fairly 
straightforward.

For the naturally parameterized \acrshort{nlp} framework, one limitation is the 
assumption that no points of nullity higher than one occur during the solution 
process 
within a cell. This effectively means that no bifurcating solution curves exist 
when a 
global parametric study is carried out. Technically, the numerical algorithm 
proposed 
in the end of the relevant Chapter is handling such bifurcating behavior when 
the 
solution transitions between cells, but this is specifically adjusted for 
parametric 
programs involving inequality constraints where loss of \acrshort{licq}, 
singularity 
of the Hessian in the tangent space, or both occur during the transition. 
Furthermore, 
the smoothness assumptions imposed on the \acrshort{tpe} and the constraints, 
while 
realistic for most mechanics applications, restrict the application of the 
parametric optimization framework to problems that do not involve severe 
discontinuities and, therefore, can be modelled using smoothness assumptions.

Lastly, regarding the weighted least squares predictor, a basic limitation is 
the fact 
that a least squares process does not result in a consistent prediction scheme. 
This 
means that as $\Delta s\rightarrow 0$, the prediction yielded by any 
\acrshort{wls} 
variant does not converge asymptotically to the tangent vector at the current 
point. 
In practice however, this is not an issue since the proposed prediction scheme 
is 
meant to provide stability and improve convergence performance for problems 
where a 
relatively large step-length is used. 


\section{Future Directions}
%%%%%%%%% HYBRID BEAM
This investigation has been a first unique attempt to establish a parametric 
nonlinear 
programming framework for the general parametric study of planar frames using a 
high 
performance, hybrid, \acrshort{nlp}-based element. There are, therefore, many 
parts 
that should be examined further and features to be added in future research 
efforts. 
Below we list some recommendations:

\begin{itemize}
	\item For the hybrid beam element,  the use of different nonlinear 
	programming 
	solution schemes can further be investigated since the optimization 
	literature, 
	which we adhere to emphasize throughout this study, possesses a quite rich 
	collection of solution techniques.  Some other directions include the 
	formulation  
	to support initially curved elements, dynamic effects, and three 
	dimensional settings that necessitate special treatment of the 
	non-commutative 
	rotational field. 
	
	\item With regards to the section state determination phase, the element 
	capability can be extended to address composite section behavior, such that 
	of a 
	reinforced concrete section. In addition, the constrained stress space 
	return 
	mapping algorithm shown for the von Mises criterion can be extended for 
	other 
	materials. Lastly, modifying the relevant rate expressions to account for 
	the 
	effect of nonlinear kinematic hardening is also an additional feature that 
	can be 
	incorporated.
	
	\item While parameterization with respect to a natural parameter is crucial 
	for 
	the \acrshort{pnlp} framework proposed in this work, examining artificial 
	embeddings, especially in conjunction with appropriate \acrshort{nlp} 
	solution 
	methods, might also yield valuable results, especially if intermediate 
	operating 
	states of the structural system are of no particular interest. Another 
	important 
	extension to the present \acrshort{npnlp} framework is bifurcation tracking 
	and 
	branch switching capabilities, which have not been investigated in this 
	work.
	
	\item Lastly, more research is required in order to better understand the 
	impact 
	of the polynomial order, number of previously converged points, the 
	weighting 
	approach and how these aspects of the \acrshort{wls} predictor interact 
	with each 
	other and with the underlying problem. Furthermore, it might prove valuable 
	to 
	explore the additional options that the \acrshort{wls} framework provides 
	with 
	respect to other aspects of predictor-corrector algorithms and, in 
	particular, the 
	step-length adjustment phase. Another direction could be the investigation 
	on the 
	influence of arc-length measurements on the quality of the local 
	approximation.
	
\end{itemize}


