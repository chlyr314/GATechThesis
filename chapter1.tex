\chapter{Introduction and Background}

\section{Literature Review}

\subsection{Beam Element Technologies}
Engineering applications involving structures undergoing large
displacements, rotations and elastic or inelastic deformations are quite
common in several fields and industries. As far as rod-like structures
are concerned, computational extensions of the classical Euler-Bernoulli theory
are required for the accurate prediction of the aforementioned effects, but
also to account for shear flexibility considerations.
In small displacement theory, the undeformed and deformed configurations are
regarded as identical and, therefore, only one configuration acts as
referential. Yet, when large displacements are considered, the
description of beam kinematics and equilibrium have to take into account that
the deformed geometry of the beam can be drastically different from the
undeformed one.  Within the Lagrangian framework, one chooses a body configuration
to refer to, giving rise to the notion of total and updated
Lagrangian formulations \cite{Bathe:1979}. In the former, the initial (or 
undeformed)
configuration is chosen as a reference, whereas in the latter, typically
associated with path-dependent problems, one refers to the current (or deformed)
one.

Several methods have been employed over the years to tackle geometrically
nonlinear problems of beams within the context of finite elements.  One approach
is to introduce a local and rigid Cartesian coordinate frame associated with
each element and rotating with it. Small displacement assumptions can then be
applied locally and computed quantities are referred back to the global
frame by the associated element transformation matrix. Such approaches, as
typically encountered in the treatment of large displacements of beams
and shells, are often termed co-rotational 
\cite{Belytschko:1973,Belytschko:1977,Rankin,Crisfield2},
are associated with small strain assumptions and ordinarily require fine
discretization. Early application of such approaches
are found in the works of Argyris et.al. \cite{Argyris} and Wempner 
\cite{Wempner:1969}. For
a survey on co-rotational methods the interested reader may refer to Felippa
and Haugen \cite{Felippa:2005}.

Another approach, which we follow in this work, considers orthogonal coordinate
frames attached to material points along the beam centerline. By choosing
a reference configuration the task then is to trace the current position vectors
of such frames. Since beam cross-sections are assigned to material points of the
centerline, these frames completely describe the current beam configuration 
once the transformation of the frames from the current to the reference
configuration is established. Formulations adhering to this geometric description
have often been termed geometrically-exact and have been considered by
Reissner \cite{Reissner1}, and further
extended in the works of several authors, such as Simo \cite{Simo1}, Simo and
Vu-Quoc \cite{Simo2,Simo3}, Cardona and Geradin \cite{Cardona1}, Ibrahimbregovic 
\cite{Ibrahim1},  Romero and Armero \cite{RomArm}, among others.
The geometrically-exact formulation is particularly attractive because the
kinematic description is expressed directly in a global inertial frame for all
elements, as opposed to local element-specific frames in the co-rotational method. For 
three-dimensional problems, the description of finite rotations requires
special treatment (e.g., see \cite{Argyris:1982,Romero:2004}) due to their
non-commutative algebraic character. There are, however, methods that
are able to tackle large rotation problems without considering nodal rotational
variables (e.g., see Romero \cite{Romero:2008}, Shabana and Yakoub 
\cite{Shabana}).
For a comparison between co-rotational and geometrically-exact approaches,
see Mathisen et al. \cite{Mathisen}.
More recent developments of geometrically-exact models in the context of
finite element analysis include viscous damping extensions\cite{Oliveto},
incorporation of the Green-Lagrange tensor as strain measure \cite{Panteli},
a mixed formulation based on complementary energy principles \cite{Santos},
and modeling of fiber-based
materials\cite{Meier}, to name but a few. Marino \cite{Marino} and
Tasora et al. \cite{Tasora} also implemented geometrically-exact formulations
in the framework of isogeometric analysis \cite{HughesBaz}, while
Salehi and Sideris\cite{Salehi}, following Reissner's theory, developed a
large-strain force-based formulation capable of tackling material softening.

Relevant approaches where strain measures are included as nodal variables are often
termed strain or deformation-based and, among others, have been
investigated by Planinc
et al. \cite{Planinc} for the plane case of a geometrically-exact beam, in
order to properly account for the effect of local instabilities on the tangent
stiffness matrix due to inelastic response. Extensions to two-dimensional
dynamic cases \cite{Gems} and to three-dimensional models \cite{Zupan}
handling issues related to strain-objectivity have also been
achieved. A similar formulation has also been presented by Saje \cite{Saje1},
where the exact kinematic relations are incorporated using the Hu-Washizu variational
principle and nodal unknowns comprise only rotational variables. Bayo et al.
\cite{Bayo} introduced a penalty formulation in order to take into account
kinematic and motion constraints. This approach avoids the use of additional
Lagrange multiplier unknowns, yet the penalized constraints are not
satisfied exactly and criteria for the proper adjustment of the penalty
coefficients are difficult to establish.

\subsection{Modelling Inelastic Behavior in Beams}

Modelling the inelastic response of structural beams is of considerable
interest in many fields of engineering practice. While accuracy of
numerical approximations is retained even in nonlinear elastic 
simulations, this is not the case when inelastic response is expected. The 
localized and complicated nature of processes at the material level 
becomes more difficult to capture with numerical models developed at a 
higher abstraction level, such as beams. Since candidate locations for 
plastification in frame structures are usually known a priori, 
analytical models have been developed based on macroscopic, geometric and 
constitutive, assumptions that can account for the localized character of
elastoplastic response. These phenomenological formulations are commonly known
as concentrated or lumped plasticity models and have 
enjoyed widespread use, especially for reinforced concrete applications, 
while also offering low computational costs.  

An alternative approach is based on discretization of the beam cross-section
into an number of fibers or layers. This allows for the employment of 
constitutive laws directly at the material level. The element 
stiffness and internal nodal forces are then determined by integrating 
the elastoplastic rate equations over selected cross-sections in the element
interior, which are encapsulated by the quadrature points. In addition, 
this enables the
modelling of plastic zone spread over the element, thus allowing for a more 
faithful representation of the response. Elements formulated this way
are called fibre or distributed plasticity beam
elements\cite{Kaba1984,Zeris1988,Taucer1991,Spacone1992,Scordelis1984} and
have seen active development for almost 30 years. In the simplest case where
only uniaxial constitutive law is considered at the level of fibers, these
elements can adequately reproduce inelastic axial-flexural interactions in 
slender members. 

While incorporation of shear effects into the fiber constitutive law is
straightforward, reproducing an accurate macroscopic behavior by integrating
a multiaxial law over the section fibers has proven intricate in general. 
One reason for
this is the nonlocal effect of shear, which depends on structure-level 
characteristics, such loading and boundary conditions\cite{Drucker1956}.
Additional difficulties are induced when modelling complex cross-section
configurations comprised of composite materials, such as reinforced 
concrete (RC) sections. Restricting ourselves to a
review of methodologies developed in the context of fibre
elements, early studies\cite{Vecchio1986,Vecchio1988} modelled
shear-axial-flexure interaction in planar beams by iteratively enforcing 
static equilibrium for each fiber between two controlling sections. More
recently this was extended to 3D beams\cite{Bairan2007}. Other works 
introduce a stress-resultant model for shear and retain an independent 
representation axial-flexure coupling through fiber 
discretization\cite{Ranzo1998,Martino2000,Marini2006}. Taylor et
al.\cite{Taylor2003} and more recently Lyritsakis et al.\cite{Lyritsakis2021}
considered uniaxial laws on each fiber for shear and axial stresses 
separately, thereby ignoring shear-axial coupling. A more physically
rigorous approach is adopted by a number of
authors\cite{Papachristidis2010,Saritas2009,Ceresa2009,Gregori2007,Kagermanov2017},
whereby a multiaxial model is utilized at the fiber level. However, this
requires additional corrective iterations per
fiber\cite{Klinkel2002,Dodds1987,DeSouza2011} during a plastic 
step in order to enforce the zero transverse normal stress condition,
$\sigma_{yy}=\sigma_{zz}=0$, thereby introducing additional overhead.

\subsection{Local and Global Solution Methods}

When dealing with problems in optimization, some important questions that arise 
are concerned with the objective function properties and whether a particular 
algorithm can converge to a stationary point, given an initial guess. In a 
similar fashion, the problem of solving systems of nonlinear algebraic 
equations, which is encountered in virtually all branches of engineering 
practice, poses similar challenges. For problems where an underlying 
variational structure exists, the two cases above can be viewed as equivalent. 
For mechanics applications in particular, the solution of the equilibrium 
equations implies the stationarity of the total potential energy of the system. 
For such applications, it is often the case that more solutions exist for a 
particular external input, whether it is an applied load or a design parameter. 
In such cases, it is important for the underlying mathematical formulation to 
provide these tools that would allow the analysis to reveal all such solutions.

One of the
most common solution methods used for such problems is the Newton-Raphson 
method\cite{Ypma:1995}, which
converges with quadratic rates if the initial guess is in the proximity of the
solution. In this sense, the method is said to be local. It is also well known 
that it fails at singular points because the Jacobian becomes 
non-invertible\cite{Dennis:1996}. A family of 
algorithms known as quasi-Newton
methods\cite{Broyden:1965,Davidon:1991,Berndt:1974}, which are variants of the 
Newton-Raphson method, attempt to bypass the singularity issue by approximating 
the Jacobian according to certain criteria while also offering substantial
advantages as far as computational cost is concerned. Such methods have been 
utilized extensively in
computational mechanics literature\cite{Matthies:1979,Hughes:1983}, with the 
BFGS\cite{Liu:1989} variant being the most widely used in implementations.

To overcome the shortcomings of Newton-type iterative schemes, 
globally convergent methods have been an active field of research for many
decades\cite{Allgower:2003}. These are often termed continuation, imbedding or 
homotopy methods\cite{Rheinboldt:2000} and aim to converge to one or more 
solutions, irrespectively of how good the initial guess is, by tracing a smooth
set of points(or path), with both the solution and the initial guess being
members of it. Modern homotopy continuation literature within the applied
mathematics community is quite
rich\cite{Keller:1978,Li:1980,Chow:1978,Chow:1979,Watson:1989,Watson:1990,Allgower:1981,Rheinboldt:1980,Rheinboldt:1983,Wayburn:1987}
and its origins can be traced back to the works of
Davidenko\cite{Davidenko:1953a,Davidenko:1953},
Klopfenstein\cite{Klopfenstein:1961},
Haselgrove\cite{Haselgrove:1961} and Yakovlev\cite{Yakovlev:1964}, among others.
The main idea is to imbed the problem in a higher dimensional space using a
homotopy function and, if 
certain regularity conditions hold, the implicit function theorem guarantees that a
smooth parameterizable path exists that connects an arbitrary initial point 
to at least 
one solution. By construction, the Jacobian in the augmented space is always of
full rank and turning points now pose no difficulty during the numerical
treatment of the problem.

In engineering applications, and more specifically in nonlinear elasticity, the 
various so-called incremental methods which were developed independently to tackle 
problems 
of elastic stability and to trace the load-displacement curve past critical
points\cite{Wempner:1971,Bergan:1978,Bergan:1978b,Batoz:1979,Riks:1979,Ramm:1981,Crisfield3}
are closely related to the homotopy continuation framework\cite{Rheinboldt:1983}. 
Dynamic Relaxation,
which is another family of
methods that has been applied to the solution of nonlinear quasi-static
problems\cite{Cassell:1970,Brew:1971,Stricklin:1971,Papadrakakis:1981,Park:1982},
can also be regarded as a numerical application of an underlying homotopy
imbedding\cite{Oden:1973}. Explicit use of the homotopy framework in structural
mechanics investigations can be found in Watson et 
al\cite{Watson:1981report,Watson:1985},
Rheinboldt\cite{Rheinboldt:1981}. Other disciplines that have employed numerical
continuation in applications include
optimization\cite{Poore:1988,Watson:1989,Zhenghua:1996,Hillermeier:2001,Song:2008},
chemical engineering\cite{Byrne:1985,Lin:1987,Kovach:1987}, transfer problems
in electrical circuits\cite{Ushida:1984} and, more recently, problems with 
variational structure and energy
minimization\cite{Hughes:2013,Mehta:2015,Mobahi:2015}, structural
mechanics\cite{Bilasse:2009,Banerjee:2013,Ligursky:2014,Sideris:2017,Groh:2018} and 
topology
optimization\cite{Chakraborty:2019}, constitutive modelling\cite{Tari:2017}, fluid 
mechanics\cite{Yu:2016,Brown:2016}
, economics \& operations
research\cite{Borkovsky:2010,Besanko:2010,Herings:2010} and numerical quadrature
investigations for isogeometric analysis\cite{Bartovn:2016,Barendrecht:2018}, among 
others. A survey on the 
application of homotopy continuation to engineering problems is provided by 
Seydel and Hlavacek\cite{Seydel:1987}.


\section{Research Objectives}
The objective of this research is to develop a high-performance, geometrically-exact, 
hybrid beam element that is formulated in terms of nonlinear programming principles, 
whereby the total potential energy of the structure is treaded as the objective 
function. The formulation will offer the capability to model a structural member with 
just one element in both linear and nonlinear response spectrum. Since the element 
formulation is recast using optimization concepts, we aim to also develop a parametric 
nonlinear programming framework for the hybrid beam element, which would provide the 
necessary tools for general purpose parametric analyses with global convergence 
characteristics. In addition, we aim to develop a fast and robust prediction scheme to 
be used during the numerical solution of the parametric nonlinear program. 

More specifically, this study proposes:
\begin{itemize}
	\item The development of a beam element that utilizes exact kinematics and is 
	based on nonlinear programming principles. The element would be able to 
	capture 
	arbitrarily large displacements and rotations, as well as curvature localization 
	during inelastic response, with just one element per member. This is a 
	unique capability which, to our knowledge, no beam implementation possesses 
	in commercial and academic software. In addition, the element
	incorporates shear deformation effects without suffering from locking 
	behavior. The element hybridization is carried out by incorporating the 
	exact kinematic equations in the minimization statement of the potential 
	energy by introducing a set of Lagrange multipliers.
	
	\item The formulation of a fast and robust return-mapping algorithm for the 
	numerical integration of fiber-discretized beam cross-sections. The proposed 
	algorithm is an essential part of the section state determination procedure as it 
	provides axial-shear-flexure interaction capabilities and is tailored for the 
	planar hybrid beam element. Because it does not incorporate non-active components 
	of the stress tensor, there is no need to enforce the zero transverse stress 
	constraint at the fiber level, thereby, avoids a nested local Newton iteration. In 
	addition, it requires less memory storage for the state variables. Thus, 
	the stress-update algorithm is considerably faster than current 
	formulations which are based on fibre elements and multiaxial material 
	constitutive models.
	
	\item The development of a parametric nonlinear programming framework for 
	the hybrid beam 
	element that will provide the capability to perform analyses on a structural 
	system that are of general consideration. Such analyses include but are not 
	limited to: static response to load history, design and parameter optimization, 
	contact, global sensitivity analysis. It essentially provides an application 
	interface for the study of parametric dependence of the structural system 
	configuration and how these states change under large variations of the designated 
	parameter. In addition, it establishes the conditions that ensure global 
	convergence properties and the numerical tools to implement such capabilities. By 
	utilizing principles from homotopy theory, we highlight what specific
	regularity conditions need to hold so that a unique and smooth path exists which
	connects the initial configuration state, which is considered known, with at least 
	one equilibrium state of a specified potential. 
	
	\item Incorporation of a fast and reliable prediction scheme in the global 
	solution 
	algorithm used for solving the nonlinear system of equations. Numerical 
	continuation is utilized in approximating stationary points (or, equivalently, 
	solutions) of a system if the initial guess is not in their proximity. The class 
	of algorithms most commonly used for this task are the so-called 
	predictor-corrector algorithms and are used to numerically track paths generated 
	by homotopy or parametric programming imbeddings. Prediction is a crucial and, 
	usually, computationally costly phase, as it affects the quality of the first 
	estimate of the solution we seek. A predictor based on a weighted least squares 
	process is proposed, whereby the next solution estimate is determined from a 
	locally fitted curve. It bypasses the need to rely on second derivative 
	information, as is usually the case, and avoids the penalizing constraints 
	imposed 
	by interpolatory prediction schemes. In addition, it provides reliable initial 
	estimates of sought solutions at a fraction of computational cost compared to most 
	other options. Lastly, if an iterative solver is used for the correction 
	phase, such as MINRES or the more general GMRES, then the numerical 
	continuation can, in principle, be further accelerated.
	
\end{itemize}

\section{Research Scope}
The scope of this research, based on the aforementioned objectives, is listed below:

\begin{itemize}
	\item Develop a nonlinear programming framework for a beam element, using the 
	total potential energy as the objective function. The element kinematics are 
	determined from the strain-displacement equations, which are regarded as 
	constraints acting on the optimization program. The Simo-Reissner or 
	geometrically-exact kinematic relations are used in this study, which do not 
	introduce any approximation in the magnitude of nodal displacements or rotations. 
	Shear deformation is fully acounted for but cross-section warping effects are 
	ignored. The formulation pertains to quasi-static problems.
	
	\item Develop a multiaxial constitutive model for the stress-update procedure at 
	the cross-section fiber level. The formulation assumes infinitesimal strains, 
	rate-independent, associative elastoplasticity. The von Mises yield criterion is 
	adopted in the present study, where linear kinematic and general isotropic 
	hardening are incorporated in the derivations. The numerical integration of rate 
	constitutive equations of elastoplasticity is carried out using the fully 
	implicit backward Euler method. The consistent tangent modulus for the 
	proposed 
	model is also derived.
	
	\item The parametric nonlinear programming framework assumes the total potential 
	energy of the structure and the kinematic (or other) constraints are sufficiently 
	smooth functions of the state variables. In addition, certain regularity 
	conditions, which are explained in detail in the relevant sections, are assumed 
	that guarantee no bifurcating paths occur during the solution process. 
	Furthermore, in the presence of inequality constraints, we assume that the linear 
	independence constraint qualification and the strict complementarity condition 
	hold for the whole domain, with the exception of maybe a finite number of points. 
	This allows us to establish the existence of piecewise smooth paths that 
	lead to 
	stationary points and a numerical solution algorithm to track such paths is 
	developed. The derivation of the framework heavily relies on homotopy 
	continuation principles and optimization in order to establish criteria for 
	a truly global method.
	
	\item The development of the weighted least squares predictor assumes an 
	underlying polynomial basis and that a number of previously converged solution 
	points is stored in memory. The proposed algorithm is general and can be 
	applied for arbitrary, general problems where the task is to solve systems 
	of nonlinear 
	equations by numerical continuation using a predictor-corrector approach. 
\end{itemize}
\section{Organization of the Manuscript}
The remainder of the manuscript is organized as follows: 

Chapter 2 introduces the formulation of the hybrid beam element. The exact kinematics 
utilized are briefly outlined, followed by the constrained, nonlinear programming 
statement for the element. Next, we present detailed implementation details, along 
with a block elimination technique for the Hessian that allows for solving only for 
nodal displacement state variables. The chapter concludes with a set of numerical 
examples where we demonstrate the capabilities of the proposed element.

Chapter 3 presents the multiaxial elastoplastic constitutive model adjusted for the 
stress state of a beam fiber. The chapter begins with the general form of the rate 
constitutive equations for elastoplasticity, as expressed for the three-dimensional 
problem. It then proceeds with developing a mapping procedure from that space to the 
constrained fiber stress space. The equivalent set of rate equations is presented here 
with the assumption of a von Mises yield criterion. Then, implementation details 
pertaining to the application of the fully implicit Euler scheme are detailed and the 
elastic predictor-plastic correct return mapping algorithm is outlined for the 
constrained stress-update problem. The Chapter closes with a set of numerical examples 
that test the accuracy of both the algorithm, as well as its performance when used on 
a number of elastoplastic simulation examples.

Chapter 4 is concerned with the development of the parametric nonlinear programming 
framework for the hybrid beam element. The Chapter starts with a brief but detailed 
outline of homotopy continuation theory as applied to solving systems of equations, as 
well as its extension treating parametric and global optimization problems. The 
next 
section introduces a naturally parameterized nonlinear programming framework 
suited 
for structural mechanics applications that use the proposed beam element. The 
conditions that guarantee global convergence are stated and an appropriate 
predictor-corrector continuation algorithm is proposed that can track piecewise smooth 
paths. The Chapter concludes with a set of examples involving optimization of a 
function, a study of global sensitivity of a structural system to changes in axial 
stiffness of a member, as well as a frictionless contact problem using a nonlinear 
elastic constitutive law.

Chapter 5 develops the weighted least squares predictor for use in the 
predictor-corrector algorithm proposed in the previous chapter. The Chapter first 
provides a summary on the use of numerical continuation and, specifically, the role of 
the prediction phase in it. After that the main idea of a locally weighted 
least squares 
fitting using a user-defined number of previously converged solution points is 
presented, along with various additional features provided by this prediction scheme, 
such as the weighting function, stabilization and the different predictor variants 
that arise from the proposed algorithm. The following section is concerned with 
implementation details and, finally, the Chapter concludes with a set of numerical 
examples that test the performance of the proposed predictor with i) the Euler 
predictor and ii) two-step and three-step Adams-Bashforth predictors. 

Lastly, Chapter 6 contains a summary of the present research work, a discussion of its 
findings, advantages, and its limitations and suggestions regarding potential 
future 
research directions.

 
Unless stated otherwise, boldface lowercase letters indicate second order 
tensor. Wherever 
the same letters appear with an overhead arrow, this indicates the vector 
notation for the tensor represented with that letter.
In addition, any boldface letter, lowercase or uppercase, with an overhead 
arrow indicates a vector object. In constrast, matrices are represented 
exclusively by uppercase, non-italic or calligraphic, boldface letters,
whereas normal font letters indicate scalar quantities.  For more details 
regarding the convention used, see Appendix \ref{appendix:0}.





