\chapter{Introduction and Background}

\section{Literature Review}

\subsection{Beam element technologies}
Engineering applications involving structures undergoing large
displacements, rotations and elastic or inelastic deformations are quite
common in several fields and industries. As far as rod-like structures
are concerned, computational extensions of the classical Euler-Bernoulli theory
are required for the accurate prediction of the aforementioned effects, but
also to account for shear flexibility considerations.
In small displacement theory, the undeformed and deformed configurations are
regarded as identical and, therefore, only one configuration acts as
referential. Yet, when large displacements are considered, the
description of beam kinematics and equilibrium have to take into account that
the deformed geometry of the beam can be drastically different from the
undeformed one.  Within the Lagrangian framework, one chooses a body configuration
to refer to, giving rise to the notion of total and updated
Lagrangian formulations\cite{Bathe:1979}. In the former, the initial (or undeformed)
configuration is chosen as a reference, whereas in the latter, typically
associated with path-dependent problems, one refers to the current (or deformed)
one.

Several methods have been employed over the years to tackle geometrically
nonlinear problems of beams within the context of finite elements.  One approach
is to introduce a local and rigid Cartesian coordinate frame associated with
each element and rotating with it. Small displacement assumptions can then be
applied locally and computed quantities are referred back to the global
frame by the associated element transformation matrix. Such approaches, as
typically encountered in the treatment of large displacements of beams
and shells, are often termed co-rotational 
\cite{Belytschko:1973,Belytschko:1977,Rankin,Crisfield2},
are associated with small strain assumptions and ordinarily require fine
discretization. Early application of such approaches
are found in the works of Argyris et.al. \cite{Argyris} and Wempner 
\cite{Wempner:1969}. For
a survey on co-rotational methods the interested reader may refer to Felippa
and Haugen \cite{Felippa:2005}.

Another approach, which we follow in this work, considers orthogonal coordinate
frames attached to material points along the beam centerline. By choosing
a reference configuration the task then is to trace the current position vectors
of such frames. Since beam cross-sections are assigned to material points of the
centerline, these frames completely describe the current beam configuration 
once the transformation of the frames from the current to the reference
configuration is established. Formulations adhering to this geometric description
have often been termed geometrically-exact and have been considered by
Reissner \cite{Reissner1}, and further
extended in the works of several authors, such as Simo \cite{Simo1}, Simo and
Vu-Quoc \cite{Simo2,Simo3}, Cardona and Geradin \cite{Cardona1}, Ibrahimbregovic 
\cite{Ibrahim1},  Romero and Armero \cite{RomArm}, among others.
The geometrically-exact formulation is particularly attractive because the
kinematic description is expressed directly in a global inertial frame for all
elements, as opposed to local element-specific frames in the co-rotational method. For 
three-dimensional problems, the description of finite rotations requires
special treatment (e.g. see \cite{Argyris:1982,Romero:2004}) due to their
non-commutative algebraic character. There are, however, methods that
are able to tackle large rotation problems without considering nodal rotational
variables (e.g. see Romero \cite{Romero:2008}, Shabana and Yakoub 
\cite{Shabana}).
For a comparison between co-rotational and geometrically-exact approaches,
see Mathisen et al. \cite{Mathisen}.
More recent developments of geometrically-exact models in the context of
finite element analysis include viscous damping extensions\cite{Oliveto},
incorporation of the Green-Lagrange tensor as strain measure \cite{Panteli},
a mixed formulation based on complementary energy principles \cite{Santos},
and modeling of fiber-based
materials\cite{Meier}, to name but a few. Marino \cite{Marino} and
Tasora et al. \cite{Tasora} also implemented geometrically-exact formulations
in the framework of isogeometric analysis \cite{HughesBaz}, while
Salehi and Sideris\cite{Salehi}, following Reissner's theory, developed a
large-strain force-based formulation capable of tackling material softening.

Relevant approaches where strain measures are included as nodal variables are often
termed strain or deformation-based and, among others, have been
investigated by Planinc
et al. \cite{Planinc} for the plane case of a geometrically-exact beam, in
order to properly account for the effect of local instabilities on the tangent
stiffness matrix due to inelastic response. Extensions to two-dimensional
dynamic cases \cite{Gems} and to three-dimensional models \cite{Zupan}
handling issues related to strain-objectivity have also been
achieved. A similar formulation has also been presented by Saje \cite{Saje1},
where the exact kinematic relations are incorporated using the Hu-Washizu variational
principle and nodal unknowns comprise only rotational variables. Bayo et al.
\cite{Bayo} introduced a penalty formulation in order to take into account
kinematic and motion constraints. This approach avoids the use of additional
Lagrange multiplier unknowns, yet the penalized constraints are not
satisfied exactly and criteria for the proper adjustment of the penalty
coefficients are difficult to establish.

\subsection{Inelastic behavior in beams}

Modelling the inelastic response of structural beams is of considerable
interest in many fields of engineering practice. While accuracy of
numerical approximations is retained even in nonlinear elastic 
simulations, this is not the case when inelastic response is expected. The 
localized and complicated nature of processes at the material level 
becomes more difficult to capture with numerical models developed at a 
higher abstraction level, such as beams. Since candidate locations for 
plastification in frame structures are usually known a priori, 
analytical models have been developed based on macroscopic, geometric and 
constitutive, assumptions that can account for the localized character of
elastoplastic response. These phenomenological formulations are commonly known
as concentrated or lumped plasticity models and have 
enjoyed widespread use, especially for reinforced concrete applications, 
while also offering low computational costs.  

An alternative approach is based on discretization of the beam cross-section
into an number of fibers or layers. This allows for the employment of 
constitutive laws directly at the material level. The element 
stiffness and internal nodal forces are then determined by integrating 
the elastoplastic rate equations over selected cross-sections in the element
interior, which are encapsulated by the quadrature points. In addition, 
this enables the
modelling of plastic zone spread over the element, thus allowing for a more 
faithful representation of the response. Elements formulated this way
are called fibre or distributed plasticity beam
elements\cite{Kaba1984,Zeris1988,Taucer1991,Spacone1992,Scordelis1984} and
have seen active development for almost 30 years. In the simplest case where
only uniaxial constitutive law is considered at the level of fibers, these
elements can adequately reproduce inelastic axial-flexural interactions in 
slender members. 

While incorporation of shear effects into the fiber constitutive law is
straightforward, reproducing an accurate macroscopic behavior by integrating
a multiaxial law over the section fibers has proven intricate in general. 
One reason for
this is the nonlocal effect of shear, which depends on structure-level 
characteristics, such loading and boundary conditions\cite{Drucker1956}.
Additional difficulties are induced when modelling complex cross section
configurations comprised of composite materials, such as reinforced 
concrete (RC) sections. Restricting ourselves to a
review of methodologies developed in the context of fibre
elements, early studies\cite{Vecchio1986,Vecchio1988} modelled
shear-axial-flexure interaction in planar beams by iteratively enforcing 
static equilibrium for each fiber between two controlling sections. More
recently this was extended to 3D beams\cite{Bairan2007}. Other works 
introduce a stress-resultant model for shear and retain an independent 
representation axial-flexure coupling through fiber 
discretization\cite{Ranzo1998,Martino2000,Marini2006}. Taylor et
al.\cite{Taylor2003} and more recently Lyritsakis et al.\cite{Lyritsakis2021}
considered uniaxial laws on each fiber for shear and axial stresses 
separately, thereby ignoring coupling shear-axial coupling. A more physically
rigorous approach is adopted by a number of
authors\cite{Papachristidis2010,Saritas2009,Ceresa2009,Gregori2007,Kagermanov2017},
whereby a multiaxial model is utilized at the fiber level. However, this
requires additional corrective iterations per
fiber\cite{Klinkel2002,Dodds1987,DeSouza2011} during a plastic 
step in order to enforce the zero transverse normal stress condition,
$\sigma_{yy}=\sigma_{zz}=0$, thereby inducing additional overhead.

\section{Research Objectives}

\section{Research Scope}

\section{Organization of the Manuscript}

\section{Notation}
Unless state otherwise, boldface lowercase letters indicate second order 
tensor. Wherever 
the same letters appear with an overhead arrow, this indicates the vector 
notation for the tensor represented with that letter. The convention adopted 
herein for storing second order tensors in a vector object is the following:
\begin{equation*}
	\boldsymbol{\sigma} = \begin{bmatrix}
		\sigma_{11} & \sigma_{12} & \sigma_{13} \\
		\sigma_{12} & \sigma_{22} & \sigma_{23} \\
		\sigma_{13} & \sigma_{23} & \sigma_{33} \\
	\end{bmatrix}\rightarrow \bvec{\sigma} = \begin{bmatrix}
		\sigma_{11} & \sigma_{22} & \sigma_{33} & \sigma_{12} & \sigma_{23} & 
		\sigma_{13}
	\end{bmatrix}^T
\end{equation*}

For strain tensors, components for which $i\neq j$ are multiplied by 2.

In addition, any boldface letter, lowercase or uppercase, with an overhead 
arrow indicates a vector object. In constrast, matrices are represented 
exclusively by uppercase, non-italic, boldface letters. Finally, normal font 
letters indicate scalar quantities.